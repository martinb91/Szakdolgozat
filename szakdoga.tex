\documentclass[11pt]{article}
\usepackage[utf8]{inputenc}
\usepackage{t1enc}
\def\magyarOptions{defaults=hu-min}

\begin{document}

A rendszer segít eligazodni a fesztiválok világában, és az érdeklődőknek eldönteni, hogy melyik fesztiválon szeretne részt venni
A rendszer funkciói(back-end):\\
Stílusok felvitele, módosítása (mely értékkészletből, majd hozzá lehet adni a zenekarokhoz, és a fesztiválokhoz)\\
Zenekarok felvitele, módosítása, esetleges archiválása\\
Fesztiválok felvitele, módosítása, a meglévő fesztiválhoz koncert rendelése(fellépő + időpont)\\
A fesztiválokhoz szállások és jegyek felvitele(jövőben ha lesz rá idő.)\\
Zenekarok keresése és listázása név vagy stílus alapján, ugyan ez fesztiválokra is + fesztiváloknál még helyi, regionális keresés is lehetséges.\\

front-end(van backend része is):
A lekérdezett zenekarra rákattintva visszatér a zenekar adataival és a letárolt fellépéseit is megmutatja kronológia vagy relevancia(egyéb is lehet priorítás) szerint.
Fesztiválra kattintva dobja a fellépőket, melyek megtekinthetők hyperlink segítségével.(Azért lehet hasznos, ha esetleg nem ismerné a felhasználó a zenekart, vagy ha kíváncsi, hogy hol lesz még fellépése).

Konkrét interfészek(b.e. szempontjából):
\\- amin befelé jön több stílus(letárolás) illetve egy zenekar amelyiket letároljuk és felszereljük az adott stílusokat (lehet create vagy modify)
\\- amin befelé jön egy stílus(letárolás)
\\- ami stílus szerint adja vissza a zenekarokat.
\\- ami a zenekarok listáját visszaadja
\\- ami a fesztiválok listáját visszaadja
\\- ami név szerint ad vissza egy fesztivált
\\- ami név szerint ad vissza egy zenekart
\\- amin befelé jön egy fesztivál és zenekarok listája, hozzá fűzve egy időpont is(amikor fel fog lépni(ehhez külön osztály és db-ben is változás lesz)).
 
????
kép hogy megy át json-ben.(elérési utat adok át?)



\end{document}