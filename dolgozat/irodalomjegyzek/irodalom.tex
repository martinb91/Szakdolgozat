\begin{thebibliography}{x}
\addcontentsline{toc}{chapter}{\bibname}

% TODO: Angular-ral, JavaScript-el kapcsolatos irodalmak.

% TODO: Java alapkönyv, keretrendszerekről néhány, Spring

% TODO: A fesztiválkereső alkalmazások webcímei

[1]\url{ http://mfdesign.hu/cikkek/reszponziv_design } %idézet

[2]\url{ https://martinfowler.com/articles/injection.html }

[3]\url{ https://spring.io/ }

[4]\url{ https://angular.io/ }

[5]\url{ https://github.com/minutuslausus }

[6]\url{ http://sanfranciscoboljottem.com/ }

[7]\url{ https://docs.oracle.com/javaee/6/tutorial/doc/docinfo.html }

[8]\url{ https://www.ibm.com/developerworks/webservices/library/ws-restful/ }

[9]\url{ https://stackoverflow.com/questions/10604298/spring-component-versus-bean }

[10]\url{ http://nyelvek.inf.elte.hu/leirasok/TypeScript/index.php?chapter=1 }

[11] Dr. Kovács László: Adatbázis Rendszerek I.

[12] Dr. Kovács László, Dr. Pance Miklós: Adatmodellezés és adatkezelési technikák (2011)

[13]\url{ http://www.memooc.hu/ }

[14] Antal Margit: Java alapú webtechnológiák, Scientia Kiadó, Kolozsvár, 2009.

[15]\url{ https://www.portfolio.hu/gazdasag/ezek-a-fesztivalok-atrendezik-magyarorszag-turisztikai-terkepet.233802.html } %cikk alapján következtetés

[16]\url{ https://turizmus.com/desztinaciok/duborog-a-fesztivalturizmus-1130113 } %cikk alapján következtetés

[17]\url{ https://nodejs.org/en/ }

[18] Ficsor Lajos, Dr. Kovács László, Krizsán Zoltán, Dr. Kusper Gábor: Szoftvertesztelés

[21]\url{https://mtu.gov.hu/documents/prod/Bulletin-2009_3.pdf } %idézet

[22] Dr. Millisits Endre HASHTAGEK ÉS VÉDJEGYOLTALOM

[23] \url{ https://docs.spring.io/spring-boot/docs/current/reference/html/getting-started-system-requirements.html }

[31] \url{ http://nodehun.blogspot.hu/2015/05/mi-az-az-npm.html } %idézet

\end{thebibliography}
