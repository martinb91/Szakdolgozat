\Chapter{Bevezetés}

% TODO: Itt kellene meggyőzni az olvasót, hogy milyen izgalmas, és fontos témáról van szó.
% A végén fixálandó

A szakdolgozat témájául egy fesztivál kereső web-es alkalmazást választottam. Azért esett erre a választásom mert, mint a korosztályom jelentős része, én is kedvelem a fesztiválok világát. Viszont gyakran előfordult, hogy nem jutott el hozzám időben a megrendezésre kerülő kisebb fesztiválok, illetve koncertek programja, így gyakran lemaradtam az általam kedvelt zenekarok fellépéseiről. A keresési igényeimnek megfelelően tervezem elkészíteni az alkalmazást. Habár léteznek hasonló weboldalak, de számomra kedvezőtlen keresési feltételeket biztosítanak. Mivel a problémakör jelentős mennyiségű személyt érint, így feltehetőleg mások számára is jelentkezett ez az információ hiány, melyet igyekszem az elkészülő szoftvertermék segítségével kiküszöbölni. 

A szakdolgozat készítésnél a következő irányelveket tűztem ki célul az elkészülő program elé.
Az általam készített rendszer segítsen eligazodni a fesztiválok világában, és az érdeklődőknek eldönteni, hogy melyik eseményen szeretne részt venni. Az alklamzás hivatott növelni a kisebb költségvetésből gazdálkodó fesztiválok versenyhelyzetét, illetve megismerteti a felhasználót a fesztiválok fellépőivel, ha azokat esetleg nem ismerné, ezzel is segítve a döntési folyamatot. Mindezt piaci profit mentesen működve kivánom elérni.

A felhasznált technológiák közül eddig még soha egyiket sem alkalmaztam a szakdolgozat készítése előtt. Így rengeteg kihívással találtam magam szembe, sőt a webes-technológiákat és alkalmazásokat sem igazán ismertem. Amikor belevágtam a projektbe akkortájt helyezkedtem el egy nagy vállalatnál és ott sikerült közelebb kerülnöm a Java EE-hez, habár csak az 1.4 verziójával, amely finoman szólva sem túl modern. Az iparban dolgozó vagy az informatikával jó barátságot ápoló ismerőseim közül többektől hallottam, hogy a Java-s világban a Spring keretrendszer eléggé közkedvelté vált manapság. Kis utána olvasással rájöttem, hogy egy ilyen alkalmazást készítek, hiszen valóban piacképesnek tűnik. A többi alkalmazott technológiát témavezetőmnek, Piller Imrének köszönhetem.  Az elkészülő program rétegzett struktúrát követ, ez lassítja a fejlesztést és a tervezést. Viszont könnyen módosithatóvá teszi az alkalmazást és detektálhatóbbá a hibákat.

A dolgozatban először végigveszünk pár már az interneten megtalálható hasonló alkalmazást, majd specifikáljuk a követelményeket. A specifikáció elkészültével a tervezési fázisban már megoldási lehetőségeket fogunk keresni az egyes felmerülő problémákra. Végül az implementációs részben bemutatok néhány érdekesebb megoldást az elkészült szoftverből, és megemlítjük a tesztelési lehetőségeket.

