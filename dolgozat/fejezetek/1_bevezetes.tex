\Chapter{Bevezetés}

A szakdolgozatom témájának egy fesztivál kereső web-es alkalmazást választottam. Azért e mellett döntöttem, mert mint a korosztályom jelentős része, én is kedvelem a fesztiválok világát. Gyakran előfordult, hogy nem jutott el hozzám időben a megrendezésre kerülő kisebb fesztiválok, illetve koncertek programja, így gyakran lemaradtam az általam kedvelt zenekarok fellépéseiről. A keresési igényeimnek megfelelően tervezem elkészíteni az alkalmazást. Léteznek hasonló weboldalak, de számomra nem biztosítanak megfelelő keresési feltételeket. Mivel a problémakör jelentős mennyiségű személyt érint, így feltehetőleg mások számára is jelentkezett ez az információ hiány, melyet igyekszem az elkészülő szoftver segítségével kiküszöbölni. 

A szakdolgozat készítésnél a következő irányelveket tűztem ki célul az elkészülő program elé.
Az általam készített rendszer segítsen eligazodni a fesztiválok világában, és az érdeklődőknek eldönteni, hogy melyik eseményen vegyen majd részt. Az alkalmazás hivatott növelni a kisebb költségvetésből gazdálkodó fesztiválok versenyhelyzetét, illetve megismerteti a felhasználót a fesztiválok fellépőivel, ha azokat esetleg nem ismerné, ezzel is segítve a döntési folyamatot. Mindezt piaci profitszerzéstől mentesen kivánom elérni.

A szakdolgozatban elkészített alkalmazásokhoz használt technológiák többségében még újak voltak számomra, így rengeteg kihívással találtam magam szembe. A dolgozat megírásával párhuzamosan tapasztalatot szereztem a jelenleg, az iparban használt módszerek terén. Az iparban Java-s területen dolgozó, illetve az informatikával jó barátságot ápoló ismerőseim a \textit{Spring} keretrendszer használatát javasolták. Áttekintve az elérhető alternatívákat, és mérlegelve az alkalmazás működéséhez szükséges funkciókat, a \textit{Spring} keretrendszer mellett döntöttem. Az ehhez kapcsolódó további webes technológiák kiválasztásában témavezetőm is segítségemre volt.

Az elkészülő program rétegzett struktúrát követ. Ennek a megtervezése körülményesebb ugyan, viszont a későbbiekben jobban áttekinthető és könnyebben tesztelhető, karbantartható struktúrát eredményez. Az esetlegesen felmerülő hibák okának megkeresése, a problémák elhárítása is egyszerűbbé válik.

A dolgozatban először áttekintem az Interneten már megtalálható, hasonló célból készített alkalmazásokat. Ezt követően specifikálom az alkalmazással szemben támasztott követelményeket. A specifikáció elkészültével a tervezési fázisban már megoldási lehetőségeket keresek az egyes felmerülő technikai problémákra. Végül, az implementációs részben bemutatok néhány érdekesebb megoldást az elkészült szoftverből, és javaslatokat teszek az alkalmazás későbbi fejlesztési, tesztelési módjára vonatkozóan.
