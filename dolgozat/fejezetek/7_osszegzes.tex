\Chapter{Összegzés}

A szakdolgozat témáját képező fesztivál kereső alkalmazás megtervezése és megvalósítása volt a feladatom. Úgy érzem, hogy ezt a terveknek megfelelően sikerült elkészítenem. Elkészült mind a szerveroldali, mind pedig a kliensoldali alkalmazás, amelyek megfelelő módon kommunikálnak egymással. Elkészült továbbá a felhasználói fiókhoz kapcsolodó funkciók közül a belépés és a regisztráció. A fellépőknél és a fesztiváloknál megvalósult az új fellépő felvitele, törlése és módosítása. Ezenkívül létrehoztam a koncertekhez és a szállásokhoz a rögzítés funkciót.

A szakdolgozatban bemutatott alkalmazás tekinthető az első nagyobb, önállóan elkészített webes alkalmazásomnak. Természetesen sok hasznos ötletet és iránymutatást kaptam a témavezetőmtől, amelyeket a lehető legjobb hatásfokkal próbáltam beépíteni az alkalmazásomba. Az is kiderült számomra, hogy mennyire tudom használni az egyetemen elsajátított ismereteimet és milyen eredményekkel tudom kombinálni ezt új ismeretekkel, melyeket önállóan kellett megszereznem. Szemléletmódbeli változást eredményezett a szakdolgozat megírása, hiszen ennek kapcsán ismerhettem meg jobban a szerver-kliens architektúrát programozás technikai szempontból. Az egyetemen oktatott tárgyak közül  különösképpen a következő tárgyak voltak a segítségemre:  Adatbázis rendszerek, Objektum Orientált Programozás, Java Technológiák, Szoftvertechnológia. Az egyetemi tárgyakon kivül sikerült megismerkednem az \textit{Angular} és \textit{Spring Boot} keretrendszerekkel, melyek további piacképes tudáshoz juttatak, és így remélhetőleg nem okoz majd gondot a munkaerőpiacon maradnom a jövőben.

Az alkalmazás további fejlesztése a jövőre nézve még rengeteg izgalmas kihivást rejt magában. Ilyen például a felhasználói fiókok jogköreinek bevezetésében rejlő kihívások. A felhasználói felület dízájnelemekkel való bővítése, szinesítése is a fejlesztés későbbi szakaszára maradt, ehhez szakember bevonására is szükség lehet, így emiatt sem valósulhatott meg az alkalmazás publikálása azonnal a szakdolgozat megírását követően. Ahogy a dolgozatomban korábban említettem, a jegyeket is szeretném majd a jövőben reprezentálni az alkalmazásban, és így árkategóriára szerint is szűrhetünk majd. Ahhoz, hogy az alkalmazás valóban sikeres lehessen, és a felmerülő igényeket fel tudjam térképezni,  valamilyen módon motiválni kell a fellépőket és a fesztivál szervezőket, hogy regisztrálják magukat a rendszerbe, és minden aktualitást közöljenek a rajongóikkal. Ahhoz, hogy egy igazán jó, széles körben használt alkalmazássá váljon az elkészült szoftver, még rengeteg további fejlesztésre lesz szükség, de összességében elégedett vagyok a dolgozat megírása során elért, és az abban bemutatott eredményekkel.

\newpage

\section*{Summary}

The aim of this work was to design and implement an application for searching beside festivals. As I see, it has completed according to the plans. The server and the client side applications are ready and they are able to communicate with each others in an appropriate way. The users can register and sign in into the system. They can add or remove festivals and performers. Furthermore, I have created functions for registering concerts and accomodations.

The presented application is my first, larger web application. Naturally, I have got many advices from my supervisor and I have try to apply them. It also have revealed for me the importance of university courses while I used the gained knowledge in combination with my own studies. It has changed my approach in the sense of programming techniques in the field of client-server communication. I would like to emphasize the importance of my previous Database Systems, Object Oriented Programming, Java Technologies and Software Technologies courses. Beyond the compulsory topics, I have become familiar with the \textit{Angular} and the \textit{Sprint Boot} frameworks, which provide me useful knowledge in the information technology industry.

The further development of the application reserves many interesting challanges. For example, I have to introduce a more sophisticated role management system. The design of the user interface also have to be improved. I think, it makes necessary to involve UI specialists into the project before the application have published for the wider audience.

As I have mentioned before, there is a plan to extend the application by managing the concert tickets. It makes possible to find and filter events by prices. For making the application successful and for investigating the requirements of the community, I have to motivate the performers and festival organizers for registering and using the web application. They have to provide actual informations for the audience. It also necessary many further improvement for resulting a popular service, but I am proud of the mentioned results.
