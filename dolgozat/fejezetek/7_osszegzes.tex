\Chapter{Összegzés}

A szakdolgozat témáját képező fesztivál kereső alkalmazás megtervezése és megvalósítása volt a feladatom. Úgy érzem, hogy ezt sikerült elkészítenem. Elkészült mind a szerveroldali  mind a kliensoldali alkalmazás, amelyek megfelelő módon kommunikálnak egymással. Elkészült a felhasználói fiókhoz kapcsolodó funkciók közül a belépés és a regisztráció. A fellépőknél és a fesztiváloknál megvalósult az új fellépő felvitele, törlése és módosítása. Ezenkívül megcsináltam a koncertekhez  és a szállásokhoz a rögzítés funkciót.

A szakdolgozat készítésének során elkészült az első webes alkalmazásom, amelyet teljesegészében én készítettem. Természetesen sok hasznos ötletet és iránymutatást kaptam a témavezetőmtől, amelyeket a lehető legjobb hatásfokkal próbáltam beépíteni az alkalmazásomba. Az is kiderült számomra, hogy mennyire tudom használni az egyetemen elsajátított ismereteimet és milyen eredményekkel tudom kombinálni ezt új ismeretekkel, melyeket önállóan kellett megszereznem. Szemléletmódbeli változást eredményezett a szakdolgozat megírása, hiszen nem ismertem a szerver-kliens architektúrát programozás technikai szempontból. Az egyetemen oktatott tárgyak közül  különösképpen a következő tárgyak voltak a segítségemre:  Adatbázis rendszerek, Objektum Orientált Programozás,  Java Technológiák, Szoftvertechnológia. Az egyetemi tárgyakon kivül sikerült megismerkednem az Angular és Spring Boot keretrendszerekkel, melyek további piacképes tudáshoz juttatak és így remélhetőleg nem okoz majd gondot a munkaerőpiacon maradnom a jövőben.

Az alkalmazás a jövőre nézve még rengeteg izgalmas kihivást rejt magában, ilyen például a felhasználóifiókok jogköreinek bevezetésében rejlő kihívások. Az alkalmazás valódi adatbázisra csatolását is tervezem. A felhasználói felület dízájnelemekkel való bővítése, szinesítése is a fejlesztés későbbi szakaszára maradt, ehhez szakember bevonására is szükség lehet, így emiatt sem valósulhatott meg a szakdolgozati fázisban. Ahogy a dolgozatban írtam a jegyeket is szeretném majd a jövőben reprezentálni az alkalmazásban és így árkategóriára is képesek lehetünk szűrni a jövőben.  Az alkalmazás valóban sikeres lehessen és a felmerülő igényeket fel tudjam térképezni, ahhoz valamilyen módon motiválni kell a fellépőket és a fesztivál szervezőket, hogy regisztrálják magukat a rendszerbe és minden aktualítást közöljenek a rajongóikkal. Tehát még rengeteg mindenen lehetne csiszolni, hogy igazán jó termék válljon az alkalmazásból, de összességében elégedett vagyok az eddig elkészült megoldásokkal.