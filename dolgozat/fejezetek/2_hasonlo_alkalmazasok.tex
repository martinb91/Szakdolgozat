\section{Fesztiválokhoz kapcsolódó szolgáltatások}

% TODO: Összeszedni lehetőleg az összes olyan funkciót, amelyik fesztiválokhoz kapcsolható lehet, mint releváns szolgáltatás/funkció!

Én a klasszikus értelembe vett zenei vagy zenei vonatkozású fesztiválokra fókuszálok.
A fesztiválokhoz nagyon változatos és sokrétű funkció csoportok tartoznak, melyeket több szempont alapján kategorizálhatunk. Szedjük ezeket szét többféleképpen, hogy meg tudjuk határozni, hogy mik legyenek a szoftvertermékünk valódi megvalósult funkciói.

A fesztiválozáshoz, mint szolgáltatáshoz közvetlenül és közvetve kapcsolódó funkciók:
Közvetlenül kapcsolódó funkciók: Amelyekért a szolgáltatást igénybe veszi a használója.
Vegyük sorra ezeket: 
\begin{itemize}
\item Fellépők, a legnagyobb vonzereje a szolgáltatásnak, ezeket kellő időben és helyen kell megismertetni a közönséggel. Ide értve nem csak a zenei produkciókat, de egyéb performance-t, és előadásokat, prevenciókat.
\item Élelmiszer és ital, a zenei fesztiválok jelentős része 3-10 napos periódusban gondolkodik. Alapvető szükségletet elégít ki, általában egy ilyen esemény alatt nagyobb összeget költenek erre a kategóriára, mint magára a belépőre, utazásra, szállásra.
\item Egyéb sport, kulturális, szellemi programok, vetélkedők. Technológiai bemutatók kipróbálási lehetőséggel.
\item VIP szolgáltatások, közönségtalálkozók.
\end{itemize}

Közvetve kapcsolódó funkciók: Amelyeket nem érzékel, amikért nem fizetne önmagában, illetve nem kell tudnia a fesztiválozónak, de kényelmetlenül érezné magát, ha nem valósulnának meg az alábbi szolgáltatások.
\begin{itemize}
\item Takarítás, WC használat, esetenként a zuhanyzási és alvási lehetőség.
\item A színpadok, dekorációk, kerítések építése, karbantartása, lebontása.
\item A fellépők oda és haza juttatása zökkenőmentesen.
\item Készletgazdálkodás, promóciók, kérdőívek.
\item Média megjelenés.
\item Elektronikus, internetes hálózat, közvilágítás.
\item Biztonság, az időjárás viszontagságai ellen védelem, értékmegőrzés.
\item Rekreáció, egészségmegőrzés.
\item Fodrász, kozmetikus a feltűnő és kreatív megjelenés érdekében.
\item Véradásért cserében valami ellen szolgáltatás.
\end{itemize}

Melyek azok a szolgáltatások, amelyek érdekelni fogják a felhasználót, szeretne előretudni róla, és ezeket egy információs rendszerben meg lehet jeleníteni: \begin{itemize}
\item Fellépők listája, mint már említettem ez az egyik legfőbb vonzerő.
\item Az extrém nem mindenhol kipróbálható sportolási, technológiai lehetőség.
\item Hasznos lehet a környékbeli étkezők elérhetősége, akár étlapja árakkal. Tudjuk, hogy a fesztiválok a magas áraikról és főleg a gyors kajáikról híresek. Érdemes lehet az olcsóbb, vagy a gasztronómia szerelmeseinek a környező elit éttermekről előre információt nyújtani.
\item Dohányboltok, illetve dohányzás. Magyarországon jelenleg a jogszabályi környezet befolyásolja a dohányáruk vásárlási helyét, erről nyújthat információt egy alkalmazás, és arról is, hogy lehetséges-e dohányozni a fesztiválterületen vagy sem.
\item Szállás, vannak olyan fesztiválok, amelyek sátorhelyet és/vagy faházat biztosítanak a közönségük számára. Emellett megjeleníthetőek a kollégiumok, apartmanok, szállodák.
\item Kutyabarátság, kutya bevihető-e a rendezvényre.
\item Gyerekbarátság, gyermekek bevihetőek-e, esetleg vannak-e számukra külön programok.
\item Jegyek, kell-e fizetnünk a fesztiválra belépésért vagy ingyenesen látogatható. Ha igen, akkor milyen jegytípusok érhetőek el.
\item Strandolási lehetőség. A zenei fesztiválok jelentős része nyáron van és vízközelben, hogy a közönségnek legyen lehetősége védekezni a meleg ellen.
\end{itemize}

\Chapter{Hasonló alkalmazások}


% TODO: Táblázatos formában a fő funkciókat összehasonlítani. (Sorokban a weboldalak, oszlopokban a funkciók, és pipálgatni, hogy melyiknél melyik van meg.)
% Ez a táblázat még hiányzik!!!

Az interneten fellelhető már egypár elterjed Fesztivál/Esemény kereső alkalmazás, vannak hiányosságaik és tartalmaznak frappáns ötleteket is. Vegyünk sorra néhányat:

\section{Fesztiválkalauz}

1. http://www.fesztivalkalauz.hu : Az első dolog amivel a felhasználó találkozik az a felhasználói felület, amelynek körülbelül a felét használta ki a fejlesztő. Jelenleg egy 14"-os kijelzőn nézve ez zavaróan kicsi. Feltehetőleg ez azért is van így, mert nagyjából 10 éves fejlesztés, és akkor még nem volt ekkora kijelző választék - sem méret, sem eszköz tekintetében - amelyeken weboldalakat jelenítettünk meg, így elég volt egy statikus méretre és betűtípusra beállítani a weboldalt. Jelenleg divatosak a reszponzív weboldalak, amelyek igyekeznek kiküszöbölni ezt a problémát.
[1] A reszponzív weboldal (RWD) egy olyan megközelítéssel tervezett weboldal, amelynek a célja az, hogy optimális megjelenést biztosítson - könnyű olvashatóság, egyszerű navigáció a lehető legkevesebb átméretezéssel és görgetéssel - a legkülönfélébb eszközökön (az asztali számítógép monitorjától egészen a mobiltelefonokig).
A viszonylag fejlett fesztiválkereső nekem tetszik, sok mindenre rálehet keresni, könnyen és átláthatóan.
Lehet település, kerület, illetve megye szerint is keresni. Típus és dátum szerinti keresés is támogatott. Sőt módunkban áll a következőhavi eseményeket is lekérdezni egy kattintással, erről ha szeretnénk, hírlevelet is kérhetünk. A szabad szavas kereső, annyira azért mégsem szabad, mert csak a címben keres a leírásban sajnos nem.

\section{Utazzithon}

2. https://www.utazzitthon.hu/program : Az oldal mint a szlogenje is mondja: "Belföldi szállás, program és látnivaló 1 helyen", szállásokat és programokat közvetít az érdeklődőknek. A felület már reszponzív tervezés eredménye, ennek köszönhetően átlátható és könnyen használható akár mobileszközről is. A keresés nagyon világos, kereshetünk régió, tájegység, város szerint és program, látnivaló típusok alapján. Az adatbázisuk programok tekintetében elég gazdag, amikor megtekintettem közel 8000 programot ajánlottak. Itt jegyet is vásárolhatunk az egyes programokra.

\section{Fesztival.eu}

3. http://www.fesztival.eu/: A weboldal régi dizájnnal készült. A keresője nagyjából hasonló funkciókat tükrözött, mint az eddigiekben megjelenő keresési lehetőségek. Sajnos az adatbázisában csak egy fesztivál volt megtalálható, amikor ott jártam, így tesztelni nem volt módom.

\section{Fesztiválnaptár}

4. http://www.fesztivalnaptar.hu : Az iranymagyarorszag.hu által működtetett weboldal. Feltehetően az utazzitthon.hu-hoz hasonlóan a szállások közvetítése a főprofilja, és ezek mellé jönnek be a fesztiválok, koncertek mint kiegészítőszolgáltatások. Szabad szavas kereséssel és kronológiai sorrend szerinti érhetőek el a programok. A szabad szavas kereső viszont keres a leírásban is, ami sokat könnyít a felhasználó számára. A felület nem reszponzív és régebbi stílusú. Nem nagy meglepetés, hogy 100-nál kevesebb esemény programját találjuk meg. Itt jelenik meg egyedül a nemzetköziesítés, a magyaron kívül angolul és németül is elérhetőek a programok, habár a programokhoz tartozó leírások csak magyarul érhetőek el.

\section{Programturizmus}

5. https://www.programturizmus.hu/ : Az utolsó keresést támogató rendszerhez érkeztünk amit találtam, és szerintem a legletisztultabb felülettel és szolgáltatásokkal. Természetesen modern weboldal lévén, platformtól és kijelző mérettől függetlenül szépen megjelenik bármilyen eszközön. Az oldal nem csak fesztiválokra, hanem szinte minden olyan lehetőséget hivatott bemutatni, ami kimozdítja a fotelból a felületen böngésző felhasználót, ez a hozzáállás a szlogenben is tükröződik - "Ne maradj otthon!". Találunk itt a vásárok, látnivalók, és gasztronómián belül, Kolbásztöltő versenytől, a Kutyakiállításon át a Régiségvásárig mindent. Persze megjelennek a már klasszikusnak mondható szállások is. A keresés a már az előzőekben megszokott lehetőségek szerint lehetséges. Viszont az "Események" mellett megjelenik három új lehetőség is, az egyik az "Ajánlat" menü, a másik az "Érdekesség", ezek valami alapján kitüntetett események, viszont az utolsó talán a legérdekesebb. Ez pedig a koordináták alapján megmutatja, hogy hol is lesznek helyileg az események egy google térképen. A jegyet itt is vásárolhatunk. Ha egy fesztivált részletesebben megnézünk, meglepőmódon a jegyen, a címen és szálláson kívül még a környékbeli étkezési lehetőségeket is tanácsolja számunkra a weboldal.

\section{Elmenyem}

5+1. http://elmenyem.hu : Az oldal nem rendelkezik fesztiválkereső résszel, ez inkább egy blog. De rengeteg aktualitással, hírekkel szolgál a jövőben megrendezendő eseményekkel kapcsolatban. 

Az itt felsorolt weboldalak természetesen a jövőben változásokon eshetnek át. Ezek a szakdolgozat írás közbeni állapotokat tükröznek.

Extra lehetőségek, amelyekkel nem éltek az itt feltüntetett weboldalak: 
SZÉP-kártya, és egyéb fizetési lehetőségek feltüntetése.

Gyerek illetve kutyabarát-e a rendezvény. Könnyen megvalósítható, mégis látványos lehetőségeket rejt magában, hiszen egy cumi, vagy egy mancs elhelyezése az esemény mellett vagy alatt, adatbázis szinten pedig csak egy boolean változó bevezetése.

Dohányzásra, alkoholfogyasztásra, korhatárra is lehetne bevezetni hasonló kis ikonokat, ezeket is egyszerű paraméterként fel lehetne szerelni.
Értékmegőrzésre, telefon töltésre van-e lehetőség. 
Ingyenes-e a rendezvény: Ez is könnyen felkeltheti az érdeklődők figyelmét. Itt is alkalmazható lenne a már jól bevált ikonos megoldás.
A nemzetköziesítés is szempont lehet, habár ezek az oldalak elsősorban hazai piacra készültek, amelyek fesztiválok nemzetközi vendégeket várnak, azoknak a marketingprogramja is külföldre pozíciónál, és saját weboldallal is rendelkeznek. Ideértve az árak több valutában való megjelenítését is a leírások szövege mellett.

Közlekedés: Érdemes lehet feltüntetni, magát a fesztivál koordinátáit, mint ezt a programturizmus.hu weboldalon láthattuk, emellett érdemes lehet parkolási lehetőségekről térképen előre informálni az odaérkezőket, egy nagyobb eseményre akár több ezer személyautóval is érkezhetnek. A buszpályaudvarról és vasútállomásról a célhoz eljutást segítő helyi járatok menetrendjét is lehetne mellékelni. Az eseménnyel szerződött taxivállalatok telefonszámait felsorolni. A gyalogos eljutást térkép segítségével megmutatni.

Szállás: Mint láthattuk, majdnem mindegyik weboldal ajánlott szállást hotelekben. De a fesztiválok klasszikus közönsége, az nem szállodákban alszik. Általában sátorban, faházakban, kollégiumokban, vagy épp ahol eléri az álom. Érdemes lenne jelezni a felhasználó felé, hogy lehet-e sátrazni, és ha igen, van-e ennek extra költsége. A kollégiumokat, faházakat is lehetne ilyen módon jelezni.

Időjárás: Ez sajnos nem jósolható hónapokkal előre, de pár héttel a fesztivál kezdete előtt ez is felkerülhetne. Hisz mint tudjuk, ezen események javarészt fedetlen vagy részben fedett helyeken zajlanak.
Visszacsatolás: A legtöbb ilyen eseményt többször megrendezik, vannak olyanok amik már 20-30 éves hagyományra tekintenek vissza. Így az értékelések, mind a szervezők, mind a szolgáltatást igénybe vevők számára hasznos információt nyújtanak.

Étkezés: Amire csak a programturizmus.hu gondolt, és alapvetően egy jó kis kiegészítő szolgáltatás, hisz fiziológiai szükségletet elégít ki.
