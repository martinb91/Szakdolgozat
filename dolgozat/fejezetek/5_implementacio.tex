\Chapter{Implementáció}

% TODO: Ide kerülhetnek a konkrét kódrészek.
% TODO: Le lehet írni, hogy mit volt egyszerűbb/komplikáltabb összerakni.

\begin{java}
\end{java}

A fesztiválok megkereséséhez használtam a legkomplexebb folyamatot. Amelyben először a felület felől meg kell vizsgálni, hogy a keresett paramétereket megadta-e a felhasználó, és csak azokat a paramétereket küldjük el a szerveroldalra . Az x és y koordinátát egy google maps API-n keresztül szerzem meg, ahol beírja a felhasználó a kívánt települést én pedig már a településhez tartozó koordinátákkal dolgozok tovább. Ha nem ír be semmit, akkor ezek az értékek üresek maradnak és a helytől való távolságot sem érdemes elküldeni a túloldalra. Amint látható a HttpParams objektumnak csak string típusú változókat lehet átadni, így castolni kellett karakterlánccá az értékeket. A dolog érdekessége, hogy a Spring oldalon ki lehet venni más típusként is, pedig a Java egy szigorúbban típusos nyelv mint az Angular.
A lekérdezésből visszaérkezett adathalmazt leképezzük a modellnek megfelelő formára és a modellekből készült tömbbel tér vissza a metódus a megjelenítési réteghez.

\begin{verbatim}
  getEventsByQuery(posX: number, posY: number, maxFromPos: number, 
  isFree: boolean, styleName: string, begin: Date, end: Date)
   : Observable<EventModel[]> {
    let prms = new HttpParams();
    if(posX !== 0 && posX != null && posX != undefined) {
      prms = prms.append('posX', String(posX));
      prms = prms.append('posY', String(posY));
      if(maxFromPos !== 0 && maxFromPos != null && maxFromPos != undefined) {
        prms = prms.append('maxFromPos', String(maxFromPos));
      }
    }

    prms = prms.append('isFree', String(isFree));

    if(styleName !== "" && styleName != null && styleName != undefined) {
      prms = prms.append('styleName', styleName);
    }
    if(begin && begin != null && begin != undefined) {
      prms = prms.append('begin', String(begin));
    }
    if(end && end != null && end != undefined) {
      prms = prms.append('end', String(end));
    }

    return this._http.get<EventModel>
    (`${environment.Spring_API_URL}/festival/query`,
     {params: prms})
      .map(data => Object.values(data).map(fest => new EventModel(fest)));

  }
\end{verbatim}