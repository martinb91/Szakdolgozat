\Chapter{Implementáció}

% TODO: Ide kerülhetnek a konkrét kódrészek.
% TODO: Le lehet írni, hogy mit volt egyszerűbb/komplikáltabb összerakni.

\begin{java}
\end{java}

A fesztiválok megkereséséhez használtam a legkomplexebb folyamatot. Amelyben először a felület felől meg kell vizsgálni, hogy a keresett paramétereket megadta-e a felhasználó, és csak azokat a paramétereket küldjük el a szerveroldalra . Az x és y koordinátát egy google maps API-n keresztül szerzem meg, ahol beírja a felhasználó a kívánt települést, a rendszer pedig már a településhez tartozó koordinátákkal dolgozik tovább. Ha nem ír be semmit, akkor ezek az értékek üresek maradnak és a helytől való távolságot sem érdemes elküldeni a túloldalra. Amint látható a HttpParams objektumnak csak string típusú változókat lehet átadni, így castolni kellett karakterlánccá az értékeket. A dolog érdekessége, hogy a Spring oldalon ki lehet venni más típusként is, pedig a Java egy szigorúbban típusos nyelv mint az Angular.
A lekérdezésből visszaérkezett adathalmazt leképezzük a modellnek megfelelő formára és a modellekből készült tömbbel tér vissza a metódus a megjelenítési réteghez.

\begin{verbatim}
  getEventsByQuery(posX: number, posY: number, maxFromPos: number, 
  isFree: boolean, styleName: string, begin: Date, end: Date)
   : Observable<EventModel[]> {
    let prms = new HttpParams();
    if(posX !== 0 && posX != null && posX != undefined) {
      prms = prms.append('posX', String(posX));
      prms = prms.append('posY', String(posY));
      if(maxFromPos !== 0 && maxFromPos != null && maxFromPos != undefined) {
        prms = prms.append('maxFromPos', String(maxFromPos));
      }
    }

    prms = prms.append('isFree', String(isFree));

    if(styleName !== "" && styleName != null && styleName != undefined) {
      prms = prms.append('styleName', styleName);
    }
    if(begin && begin != null && begin != undefined) {
      prms = prms.append('begin', String(begin));
    }
    if(end && end != null && end != undefined) {
      prms = prms.append('end', String(end));
    }

    return this._http.get<EventModel>
    (`${environment.Spring_API_URL}/festival/query`, {params: prms})
      .map(data => Object.values(data).map(fest => new EventModel(fest)));

  }
\end{verbatim}

A controller az alábbi metódussal veszi át a beérkező kérést. Amint az látható, a metódus egyik paraméterét sem kötelező átadni neki, hisz ez egy lekérdezés, jöhet kevesebb információval is kérés. Ha érkezik dátum érték azt Date formátumra kell konvertálnunk és csak ezután hívhatjuk meg a a festivalService festsByQuery metódusát.

\begin{verbatim}
	@RequestMapping(value = "/query", method = RequestMethod.GET)
	public List<FestivalDTO> festsByQuerry(
	@RequestParam(value = "styleName", required = false) String style,
										   @RequestParam(value = "isFree", required = false) boolean isFree,
										   @DateTimeFormat(iso = DateTimeFormat.ISO.DATE) 
										   @RequestParam(value = "begin", required = false) LocalDate beginDate,
										   @DateTimeFormat(iso = DateTimeFormat.ISO.DATE) 
										   @RequestParam(value = "end", required = false) LocalDate endDate,
										   @RequestParam(value = "posX", required = false) Double posX,
										   @RequestParam(value = "posY", required = false) Double posY,
										   @RequestParam(value = "maxFromPos", required = false) Double maxFromPos){
		Date begin = null;
		Date end = null;
		if (beginDate != null){
			begin = Date.from(beginDate.atStartOfDay().atZone(ZoneId.systemDefault())
			.toInstant()); }
		if (endDate != null){
			end = Date.from(endDate.atStartOfDay().atZone(ZoneId.systemDefault())
			.toInstant());}
		return festivalService.festsByQuery( style, isFree, begin, end,
		 posX, posY, maxFromPos); }
\end{verbatim}

A festsByQuery metódust a következőképpen implementáltam.

\begin{verbatim}
    public List<FestivalDTO> festsByQuery(String style, boolean isFree, Date begin, Date end, Double posX, Double posY, Double maxFromPos) {
        List<Festival> festivals = new ArrayList<>();
        if (style == null) {
            if (!isFree) {
                festivals = datesWithoutStyle(begin, end, posX, posY, maxFromPos);
            } else {
                festivals = datesWithStyle("free", begin, end, posX, posY, maxFromPos);
            }
        } else {
            if (!isFree) {
                festivals = datesWithStyle(style, begin, end, posX, posY, maxFromPos);
            } else {
                for (Festival festival : datesWithStyle(style, begin, end, posX, posY, maxFromPos)){
                    System.out.println(festival.getName());
                    for (FestivalStyle festivalStyle : festival.getStyles()){
                        if(festivalStyle.getStyle().toLowerCase().equals("free")){
                            festivals.add(festival);
                        }
                    }
                }
            }
        }

        List<FestivalDTO> festivalDTOS = new ArrayList<>();
        for (Festival festival : festivals) {
            System.out.println(festival.getName());
            festivalDTOS.add(modelMapper.map(festival, FestivalDTO.class));
        }
        return festivalDTOS;
    }
\end{verbatim}