\Chapter{Telepítés, tesztelés}

% TODO: Az alkalmazás üzembehelyezését, konfigurációját részletesen bemutatni!

% TODO: Példa néhány felhasználói tesztre. Le lehet írni, hogy mi az amit közben korrigálni kellett. A javítás módját is le lehet írni.

% TODO: Frontend teszteléséhez használható eszközök bemutatása és néhány konkrét példa a tesztekre.

% TODO: Java egységtesztek bemutatása. Például kereséshez, szűréshez, regisztrációhoz.
\section{Telepítés}

Mindkét keretrendszer független az operációs rendszerektől, tehát minden népszerű operációs rendszert használó számítógépen el tudjuk indítani az alkalmazásokat.

\subsection{Spring}

A Spring Boot 1.5.8.RELEASE kiadást használtam, ez a Java 7-es verzióját igényli - a 8 vagy újabb ajánlott - a fordításhoz. Tehát érdemes telepíteni a lehető legújabb Java-t, mind a JRE-t mind a JDK-t.

Maven segítségével töltjük le azokat a jar fájlokat amelyektől függ az alkalmazásunk. A 3.5.0 verzió volt akkor a legújabb verzió és általam használt verzió is ez. A Maven 3.5.0 szintén igényli a Java hetedik verzióját vagy újabb kiadásait. A projekt pom.xml fájljában definiált csomagok az alkalmazás függőségei, ezeket a fordítás előtt le fogja tölteni, ha nem találja meg őket az erre a célra lefoglalt mappában. Ezeknek a csomagoknak vannak további függőségeik, azokat is feloldja a Maven, amíg a függőségi fán hiányzó levelet talál, addig folytatja ezt az eljárást. Így levesz minden terhet a vállunkról.

\subsection{Angular}

[31] A front-end részhez először telepíteni kell egy Node.js-t. Az akkor elérhető legfrissebb verziót, amely a 9.10.1 volt. Ez a keretrendszer weboldaláról(nodejs.org) letölthető. Erre azért volt szükségünk, mert ebben található egy npm nevű modul. Az npm vagy Node Package Manager, mint a neve is mutatja csomagok telepítésére és karbantartására való.

Egy csomagot kétféleképpen lehet installálni az egyik a helyi telepítés és a globális telepítés. Globálisan olyan csomagokat szokás telepíteni amiket többnyire terminálból akarunk futtatni.
Mi az Angular CLI 1.7.0-t globálisan telepítettük npm segítségével, hiszen parancssorból szeretnénk indítani majd a kiszolgálást.

A következő paranccsal érhetjük ezt el, hogy a legfrissebb verziót telepítse: \texttt{npm install -g @angular/cli}
Elvileg visszafelé kompatibilis a rendszer, tehát ha a legújabb globális verzió van fent a gépen akkor a régebbi vagy azonos verziószámú projektet képes lesz futtatni. Én Angular 5.2.0-ás verziószámú projektet futtattam benne, és minden gond nélkül elindult.
Minden projekthez tartozik egy \texttt{package.json}, úgy mint a Maven-nél a \texttt{pom.xml}, ez tárolja a projekt függőségeit. Ezeket le kell tölteni, ha van olyan amelyik nincs telepítve, én ezt már lokálisan tettem meg. Terminál segítségével ezt úgy tehetjük meg, hogy benavigálunk a projekt mappájába és futtatjuk az \texttt{npm install} parancsot, jelentősebb időbe telik mire letölt mindent amire szükség lesz a projekt futtatásához. Majd ezután, ha minden rendben ment, akkor az \texttt{ng serve} parancsra elkezd futni a program. A terminálra kiírja, hogy melyik porton indult el, ez a \texttt{localhost:4200} gyári beállításon. Ha a back-end oldali alkalmazás is fut a \texttt{localhost:8080}-as porton, akkor már az adatok is vizualizálódnak.

\section{Tesztelés}

\subsection{Felhasználói felület tesztelése}
A felhasználói felületet funkcionális tesztekkel manuálisan tudtam tesztelni, hogy minden úgy működik-e, ahogy azt a rendszertől elvártam. A tesztelés másik módja az volt, hogy JSON pipe segítségével megjelenítettem az adatokat a felületen, és így valós időben tudtam követni az interakcióim hatására bekövetkező változásokat. Van lehetőség a funkcionális tesztek automatizálására a front-end oldalon, de időmből nem futotta ennek elsajátítására.

\subsection{Back-end tesztelése}

A Spring-nél szokás használni a tesztelés által irányított fejlesztést(TDD), ez azt jelenti hogy először megírjuk a teszteket és utána a kódot hozzá. Minden modul akkor áll készen, ha definiált tesztek sikeresen lefutottak. Majd jöhet a következő modul, viszont azt már úgy kell elkészítenünk, hogy a saját tesztjei is lefussanak, illetve a régebben elkészített modulok tesztjei is sikeresek maradjanak. Én nem így készítettem el a modult, csak már a kész modulokat teszteltem.

Először a back-enddel készültem el, és nem volt felhasználói felületem(UI) aminek segítségével az egyes kéréseket szimulálhattam volna, így Postman segítségével küldtem be a minta kéréseimet és ezáltal teszteltem fejlesztési időben.

Ezután jöttek az egységtesztek.
Egység teszt, ezzel egyetlen egy metódus működését tesztelhetjük, csak a  legelemibb megoldásokat tesztelhetjük vele.

Az egység tesztek után az integrációs teszteket szokták elvégezni, azt vizsgálja, hogy a modulok jól tudnak-e együttműködni.

Stresszteszteket JMeter segítségével készítettem. A stressztesztek arra használatosak, hogy kiderítsük mekkora aktivitásnak tehetjük ki az alkalmazást.