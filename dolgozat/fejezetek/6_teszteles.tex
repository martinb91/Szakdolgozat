\Chapter{Telepítés, tesztelés}

% TODO: Az alkalmazás üzembehelyezését, konfigurációját részletesen bemutatni!

% TODO: Példa néhány felhasználói tesztre. Le lehet írni, hogy mi az amit közben korrigálni kellett. A javítás módját is le lehet írni.

% TODO: Frontend teszteléséhez használható eszközök bemutatása és néhány konkrét példa a tesztekre.

% TODO: Java egységtesztek bemutatása. Például kereséshez, szűréshez, regisztrációhoz.
\section{Telepítés}
\subsection{Spring}

[23] Spring alkalmazást minimum Java 6-os fordítóval lehet lefordítani futtatható alkalmazássá. Én a Spring Boot 1.5.8.RELEASE kiadást használtam. Itt már csak a Java 7-es verziója használható - a 8 vagy újabb ajánlott - fordításra. Én a Java 8-at használtam, amikor belevágtam a fejlesztésbe, akkor még a Java 9 nem volt elérhető. Azóta már megjelent a 10 is, de én nem váltottam verziót. Alapvetően Java 8 specifikus dolgokat sem használtam, a Java újabb verzióival még nem volt időm megismerkedni.

Maven segítségével töltöttem le azokat a jar fájlokat amelyektől függ az alkalmazásunk. A 3.5.0 verzió volt akkor a legújabb verzió és általam használt verzió is ez. A Maven 3.5.0 szintén igényli a minimum a Java hetedik verzióját.

A Maven-nek az alábbi csomagokat kell letöltenie:
\begin{verbatim}
spring-boot-starter-web
spring-boot-starter-data-jpa
spring-boot-starter-test
spring-boot-devtools
ojdbc14
\end{verbatim}

% Az adatbázis még nem bíztos hogy oracle lesz, jelenleg H2-van alatta.

\subsection{Frontend}

[31] A front-end részhez először telepítettem egy Node.js-t. Az akkor elérhető legfrissebb verziót, amely a 9.10.1 volt. Ez a keretrendszer weboldaláról(nodejs.org) letölthető. Erre azért volt szükségünk, mert ebben található egy npm nevű modul. Az npm vagy Node Package Manager, mint a neve is mutatja csomagok telepítésére és karbantartására való.

Egy csomagot kétféleképpen lehet installálni az egyik a helyi telepítés és a globális telepítés. Globálisan olyan csomagokat szokás telepíteni amiket többnyire terminálból akarunk futtatni.
Mi az Angular CLI 1.7.0-t globálisan telepítettük npm segítségével, hiszen parancssorból szeretnénk indítani majd a szervert.

A következő paranccsal érhetjük ezt el, hogy a legfrissebb verziót telepítse: npm install -g @angular/cli
Elvileg visszafelé kompatibilis a rendszer, tehát ha a legújabb globális verzió van fent a gépen akkor a régebbi vagy azonos verziószámú projektet képes lesz futtatni. Én Angular 5.2.0-ás verziószámú projektet futtattam benne, és minden gond nélkül elindult.

\section{Tesztelés}

\subsection{Felhasználói felület}
A felhasználói felületet csak manuális tesztekkel tudtam tesztelni, hogy minden úgy működik-e ahogy azt a rendszertől elvártam.
