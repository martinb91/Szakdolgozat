\Chapter{Telepítés, tesztelés}

\Section{Telepítés}

Mindkét keretrendszer független az operációs rendszerektől, tehát minden népszerű operációs rendszert használó számítógépen el tudjuk indítani az alkalmazásokat. A következő szakaszokban azt mutatom be, hogy milyen verziójú szoftvereket érdemes használni az alkalmazás telepítéséhez, illetve, hogy a telepítés milyen nagyobb lépésekből áll.

\SubSection{Spring}

A \textit{Spring Boot} 1.5.8.RELEASE kiadást használtam. Ez a Java 7-es verzióját igényli -- a 8 vagy újabb ajánlott -- a fordításhoz. Tehát érdemes telepíteni a lehető legújabb Java verziót, mind a JRE, mind pedig a JDK tekintetében.

\textit{Maven} segítségével tölthetjük le azokat a \texttt{jar} kiterjesztésű fájlokat, amelyek szükségesek az alkalmazás működéséhez. A 3.5.0 verzió volt a dolgozat megírása idején a legújabb verzió, ezért ezt használtam. A \textit{Maven} 3.5.0 szintén igényli a Java hetedik verzióját (vagy annál újabb kiadásait). A projekt \texttt{pom.xml} fájljában definiált csomagok az alkalmazás függőségei, ezeket a fordítás előtt le fogja tölteni a \textit{Maven}, ha nem találja meg őket az erre a célra lefoglalt mappában. Ezeknek a csomagoknak vannak további függőségeik, azokat is feloldja. Amíg a függőségi fán hiányzó levelet talál, addig folytatja ezt az eljárást, így levesz minden terhet a fejlesztők válláról a függőségek kezelését illetően.

\SubSection{Angular}

[31] A front-end részhez először telepíteni kell a \textit{Node.js}-t. Az aktuálisan elérhető legfrissebb verzió a 9.10.1 volt. Ez a keretrendszer weboldaláról (\texttt{nodejs.org}) letölthető. Erre azért volt szükségünk, mert ebben található egy \texttt{npm} nevű modul. Az \texttt{npm} vagy \textit{Node Package Manager}, mint a neve is mutatja, csomagok telepítésére és karbantartására való.

Egy csomagot kétféleképpen lehetséges telepíteni. Az egyik mód a lokális telepítés, a másik pedig a globális. Globálisan azokat a csomagokat installáljuk amelyeket terminálból szeretnénk a későbbiekben elindítani.
Az Angular CLI 1.7.0-t globálisan telepíthetjük \texttt{npm} segítségével, hiszen parancssorból szeretnénk indítani majd a kiszolgálást.

A következő paranccsal érhetjük ezt el, hogy a legfrissebb verziót telepítse: \texttt{npm install -g @angular/cli}
Elvileg visszafelé kompatibilis a rendszer, tehát ha a legújabb globális verzió van fent a gépen akkor a régebbi vagy azonos verziószámú projektet képes lesz futtatni. Az \textit{Angular} 5.2.0-ás verziószámú projektet futtattam benne, és minden gond nélkül elindult.

Minden projekthez tartozik egy \texttt{package.json}, úgy mint a \textit{Maven} esetében a \texttt{pom.xml}. Ez tárolja a projekt függőségeit. Ezeket le kell tölteni, ha van olyan amelyik nincs telepítve. Ezt már célszerűnek tartottam lokálisan megtenni. Terminál segítségével ezt úgy tehetjük meg, hogy a projekt mappájába navigálunk, és futtatjuk az \texttt{npm install} parancsot. Jelentősebb időbe telik, mire letölt mindent amire szükség lesz a projekt futtatásához. Ezután, ha minden rendben ment, akkor az \texttt{ng serve} parancsra elkezd futni a program. A terminálra kiírja, hogy melyik porton indult el, ez a \texttt{localhost:4200} a gyári beállítások szerint. Amennyiben a back-end oldali alkalmazás is fut a \texttt{localhost:8080}-as porton, akkor már az adatok is vizualizálódnak.

\Section{Tesztelés}

\SubSection{Felhasználói felület tesztelése}

A felhasználói felületet funkcionális tesztekkel manuálisan tudtam tesztelni, hogy minden úgy működik-e, ahogy azt a rendszertől elvártam. A tesztelés másik módja az volt, hogy \textit{JSON pipe} segítségével megjelenítettem az adatokat a felületen, és így valós időben tudtam követni az interakcióim hatására bekövetkező változásokat. Van lehetőség a funkcionális tesztek automatizálására a front-end oldalon. Az \textit{Angular} ehhez számos lehetőséget biztosít, amelyre dolgozatomban részletesen már nem térnék ki.

\SubSection{Back-end tesztelése}

A \textit{Spring}-nél szokás használni a teszt vezérelt fejlesztést (\textit{Test Driven Development}). Ez azt jelenti, hogy először megírjuk a teszteket, és utána a kódot hozzá. Minden modul akkor áll készen, ha definiált tesztek sikeresen lefutottak. Ezt követően jöhet csak a következő modul, viszont azt már úgy kell elkészítenünk, hogy a saját tesztjei is lefussanak, illetve a régebben elkészített modulok tesztjei is sikeresek maradjanak. A dolgozat megírásakor csak a kész modulok tesztelésénél maradtam.

Először a back-enddel készültem el, és nem volt felhasználói felületem amelynek segítségével az egyes kéréseket elküldhettem volna. Ahhoz, hogy mégis vizsgálni tudjam a működést, a Postman segítségével küldtem be a minta kéréseimet, és ezáltal teszteltem az aktuálisan fejlesztett elemeket.

Az alkalmazást a megbízhatóság érdekében érdemes lesz a későbbiekben egységtesztekkel bővíteni. Ezekkel az adott metódusok működését tesztelhetjük, elemi egységekként.

A függetlenül már tesztelt modulok egy nagyobb egységként való, továbbra is helyes működését integrációs tesztek készítésével garantálhatjuk. Ahogy az alkalmazásban a funkciók száma gyarapodik, nyilván az ilyen jellegű tesztekre is egyre nagyobb szükség lesz majd.

Stresszteszteket \textit{JMeter} segítségével készítettem. A stressztesztek arra használatosak, hogy kiderítsük mekkora aktivitásnak tehetjük ki az alkalmazást. A tesztek olyan tekintetben csak próbálkozás jellegűek, hogy mind a várható felhasználószámra, mind pedig a rendelkezésre álló infrastruktúrára vonatkozóan csak nagyvonalú becslések állnak jelenleg rendelkezésre.
