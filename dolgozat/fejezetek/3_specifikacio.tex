\Chapter{Az alkalmazás specifikációja}

% TODO: Részletesen leírni, hogy mire készült az alkalmazás. Jogosultsági körök, bejelentkezéssel felhasználókezeléssel kapcsolatos dolgok, képernyőképekhez vázlatos jellegű ábrák.

% TODO: Be kellene mutatni néhány használati esetet is.

Mit is szeretnénk, ha tudna az alkalmazásunk?

Fesztiválkereső alkalmazásról lévén szó, így a webalkalmazásunk legfőbb funkciója a fesztiválozni vágyó tömeg számára az értékrendjük szerint releváns fesztiválok megtalálása, és ezekről a legkülönfélébb információk eljuttatása a célközönségük számára. A fesztiválok megkeresése a lehető legtöbb kritérium alapján megvalósulhasson.

Melyek is legyenek ezek a kritériumok?

Egy fesztivált talán a neve alapján a legkönnyebb azonosítani, ez azok számára könnyíti meg a dolgot, akik már tudják, mit keresnek csak extra információkat szeretnének gyűjteni a fesztiválról vagy a környezetéről (például szálláshelyek, vagy étkezési lehetőségek).

A második keresési szempont a település alapján, gyakran fontos szempont lehet, hogy ne kelljen sokat utazni a kikapcsolódásért. Főleg ha, csak egy-egy nap programja érdekli a felhasználót, akkor jó eséllyel nem fog szállást keresni, hanem még aznap haza szeretne utazni. Ezt a keresést természetesen bővíthetjük, egy adott településtől való távolság szerinti keresésre.

A következő kritérium amire szűrhetünk az a dátum. Mindenkinek az életében vannak olyan időpontok, időszakok amelyek terheltebbek, illetve kevésbé terheltek. Sokan dolgoznak külföldön és nem minden hétvégén tudják itthon tölteni. Időintervallum szerinti szűrés azon fesztiválok megtalálásában segít, amelyek akkor kerülnek megrendezésre, amikor ráérünk.

Fellépő alapú keresés, a felhasználók túlnyomó része az alapján is szelektálja a fesztiválokat, hogy ott lesznek-e a kedvenc fellépői vagy sem. Így természetesen a fellépő alapú szelekciót is meg kell valósítani. Ezt a részt úgy képzelem el, hogy a fellépőre kereshetünk rá és neki láthatjuk a fesztiválnaptárját, tehát, hogy hol és mikor lép fel.

Stílus alapján is megtalálhatjuk a fesztiváljainkat. Egyesek számára például az is fontos, hogy milyen stílusú a fesztivál. Itt a stílus alatt elég sok mindent érthetünk, például vannak valamilyen étel vagy ital köré szerveződő fesztiválok, vannak amelyek kultúra vagy művészettel kapcsolatosak, illetve a zenei fesztiválok. A stíluson belül további stílusokra, jellemzőkre oszlik egy fesztivál jellege:  milyen stílusú zenét játszanak a fesztiválon. Jellemzők alatt mit érthetünk? Állatbarátok számára fontos lehet, hogy bevihetik-e  a kis kedvenceket. A dohányzók számára a dohányzás szabadsága, a távolról érkezők számára  a sátrazás lehetősége. Ezekre a jellemzőkre utaló kulcsszavakat kell bevezetni, amelyek segítségével szintén könnyebben eligazodhat a fesztiválozni vágyó felhasználó.

Árkategória szerint is érdemes lehet szűrni. Elképzelhető, hogy valaki számára csak a teljesen ingyenes fesztiválok férnek bele a költségvetésébe, míg lehetnek olyan felhasználók is akik inkább a drágább fesztiválokat kedvelik mert ott mondjuk az belépő árából tudnak költeni a jobb higiéniai és biztonsági feltételekre.


Jogosultsági körök:

ADMIN: Az admin jogkör az a jogosultsági kör aki felviheti a rendszerbe a felületen keresztül a fesztiválokat, szállásokat, és a fellépőket. Mindezeket törölheti és módosíthatja is az adatbázisban, de természetesen a felhasználói felületen keresztül is. ADMIN jogosultságot legtöbb esetben nagyon kevés felhasználó kaphat.

Be nem lépett felhasználó: Mindent megtekinthet, de módosítani nem tudja a rendszerben lévő adatokat, ahogy új adatot sem tud felvinni, és törölni sem tudja azokat.