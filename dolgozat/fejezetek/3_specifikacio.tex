\Chapter{Az alkalmazás specifikációja}

% TODO: Részletesen leírni, hogy mire készült az alkalmazás. Jogosultsági körök, bejelentkezéssel felhasználókezeléssel kapcsolatos dolgok, képernyőképekhez vázlatos jellegű ábrák.

% TODO: Be kellene mutatni néhány használati esetet is.

Mit is szeretnénk, ha tudna az alkalmazásunk?

Fesztiválkereső alkalmazásról lévén szó, így a webalkalmazásunk legfőbb funkciója a fesztiválozni vágyó tömeg számára az értékrendjük szerint releváns fesztiválok megtalálása, és ezekről a legkülönfélébb információk eljuttatása a célközönségük számára. A fesztiválok megkeresése a lehető legtöbb kritérium alapján megvalósulhasson. Továbbá szeretnénk ajánlani szállásokat és egyéb szolgáltatásokat amikkel élhet a fesztivál területén és közvetlen közelében.

Melyek is legyenek ezek a kritériumok?

Egy fesztivált talán a neve alapján a legkönnyebb azonosítani, ez azok számára könnyíti meg a dolgot, akik már tudják, mit keresnek csak extra információkat szeretnének gyűjteni a fesztiválról vagy a környezetéről (például szálláshelyek, vagy étkezési lehetőségek).

A második keresési szempont a település alapján, gyakran fontos szempont lehet, hogy ne kelljen sokat utazni a kikapcsolódásért. Főleg ha, csak egy-egy nap programja érdekli a felhasználót, akkor jó eséllyel nem fog szállást keresni, hanem még aznap haza szeretne utazni. Ezt a keresést természetesen bővíthetjük, egy adott településtől való távolság szerinti keresésre.

A következő kritérium amire szűrhetünk az a dátum. Mindenkinek az életében vannak olyan időpontok, időszakok amelyek terheltebbek, illetve kevésbé terheltek. Sokan dolgoznak külföldön és nem minden hétvégén tudják itthon tölteni. Időintervallum szerinti szűrés azon fesztiválok megtalálásában segít, amelyek akkor kerülnek megrendezésre, amikor ráérünk.

Fellépő alapú keresés, a felhasználók túlnyomó része az alapján is szelektálja a fesztiválokat, hogy ott lesznek-e a kedvenc fellépői vagy sem. Így természetesen a fellépő alapú szelekciót is meg kell valósítani. Ezt a részt úgy képzelem el, hogy a fellépőre kereshetünk rá és neki láthatjuk a fesztiválnaptárját, tehát, hogy hol és mikor lép fel. Külön menüként valósul meg.

Stílus alapján is megtalálhatjuk a fesztiváljainkat. Egyesek számára például az is fontos, hogy milyen stílusú a fesztivál. Itt a stílus alatt elég sok mindent érthetünk, például vannak tematikus fesztiválok amelyek, valamilyen étel vagy ital köré szerveződő fesztiválok, vannak amik a kultúra vagy művészettel kapcsolatosak, illetve a zenei fesztiválok. A stíluson belül további stílusokra, jellemzőkre oszlik egy fesztivál jellege:  milyen stílusú zenét játszanak a fesztiválon. Jellemzők alatt mit érthetünk? Állatbarátok számára fontos lehet, hogy bevihetik-e  a kis kedvenceket. A dohányzók számára a dohányzás szabadsága, a távolról érkezők számára  a sátrazás lehetősége. Ezekre a jellemzőkre utaló kulcsszavakat kell bevezetni, amelyek segítségével szintén könnyebben eligazodhat a fesztiválozni vágyó felhasználó. Ezekre a kulcsszavakra kattintva leszűrjük számára a megadott jellemzőkkel rendelkezőket.

Árkategória szerint is érdemes lehet szűrni. Elképzelhető, hogy valaki számára csak a teljesen ingyenes fesztiválok férnek bele a költségvetésébe, míg lehetnek olyan felhasználók is akik inkább a drágább fesztiválokat kedvelik mert ott feltehetőleg magasabb minőségűek a szolgáltatások, akár a közönség találkozhat személyesen a fellépővel vagy a jobb higiéniai és biztonsági feltételek miatt is választhatja, mindenki döntse el maga. Ez elképzelhető, hogy nem kerül implementálásra, mert a jegyekhez külön szakértelem kell. Miért mondom ezt? Egyszerűbb esetben van 2 verzió vagy van jegy vagy nincs. Ezt egy metaadat segítségével egyszerűen kiszűrhetjük, a metaadatunk lehet az előző pontban említett fesztiváljellemző. Bevezetve az ingyenes fesztiválokra egy, free, nincsBelépő vagy ingyenes kulcsszavakat. Viszont, ha van jegy, akkor van egyszerű dolgunk, ha csak egy fajta jegyünk van, de ez jellemzően nem így van. Általánosságban elmondható, hogy van a fesztiválon eltöltött napok szerint létező, szinte összes kombináció. Szállással kérjük vagy anélkül, és ha szállással akkor milyen szállással? A legtöbb fesztiválnak vannak szerződött partnerei, szállodák vagy kollégiumok illetve helyben is megtalálhatóak a sátorhelyek, faházak, lakókocsik, és tematikától függően még kitudja milyen alvási és pihenési lehetőségekkel nem futhatunk össze. Gyakran időintervallumhoz kötődnek a jegyárak, minél korábban veszünk meg egy jegyet annál olcsóbb. Vannak VIP jegyek, melyek az extra szolgáltatást igénylők számára lehet érdekes. És még lehetnek egyéb horizontok is, amik színesíthetik ezt a palettát, mostanában divat lett a buszok szervezése, főleg a  külföldi fesztiválokra, itt ismét megjelent egy plusz jegytípus. Lehetnek egyéb kedvezmények, és kuponok is. És ezeknek a kombinációja. Lássuk be, hogy ezeket a dolgokat és a változásaikat csak akkor lehetne jól kezelni, ha valamilyen API-n megkapnánk őket. De ezt az is gátolja, hogy ahány jegy annyi forgalmazó, mindenkinek a rendszerére nem lehet felkészíteni a mi rendszerünket, így maradunk annál a megoldásnál, hogy egy linket adunk, ahol mindenki kedvére válogathat a jegyek közül.

A keresés mellett milyen szolgáltatásokat lehet még megjeleníteni a felhasználóink számára?
A teljesség igénye nélkül felsorolok párat:
- A fesztivál megtekintésekor, mutassuk meg a fesztivál közelében levő szállási lehetőségeket, éttermeket, egyéb rekreáció és a fesztiválozáshoz kapcsolható szolgáltatásokat. Én csak a szállásokkal tervezem feltölteni az adatbázisom, de természetesen ugyanúgy lehetne biciklikölcsönzőt vagy strandot is mutatni a fesztiválterület közelében. 
- A teljes fesztivál programot elérhetővé kell tenni a fesztivál oldalán. És ha valamelyik koncert fellépőjére kattintunk, akkor tudjuk elérni a hozzá tartozó profilt. Ahol elolvashatjuk a fellépőről szóló leírást, és akár a weboldalát vagy a YouTube csatornáját is elérhetjük, ahol belehallgathatunk a népszerű slágereikbe.


Jogosultsági körök:

ADMIN: Az admin az a jogosultsági kör aki felviheti a rendszerbe a felületen keresztül a fesztiválokat, szállásokat, és a fellépőket. Mindezeket törölheti és módosíthatja is az adatbázisban, de természetesen a felhasználói felületen keresztül is. ADMIN jogosultságot legtöbb esetben nagyon kevés felhasználó kaphat. Az előzetes terveim szerint egyenlőre aki hozzáférést kap a rendszerhez az ADMIN lesz. De egyenlőre csak korlátozott számú ilyen személy lesz.

USER: Az a felhasználó aki be van jelentkezve a rendszerbe, de módosítani nem tudja azt, esetleg módosítási javaslatokat tehet. Lehetősége van feliratkozni egyes zenekarokra, illetve fesztiválokra és levelet kaphat róla, ha valami módosul. Ez a felhasználó nem biztos, hogy megvalósul a fejlesztés ezen fázisában, majd az időkorláttól is függ.

Be nem lépett felhasználó: Mindent megtekinthet, de módosítani nem tudja a rendszerben lévő adatokat, ahogy új adatot sem tud felvinni, és törölni sem tudja azokat.